\documentclass{article}
\usepackage[utf8]{inputenc}

\usepackage[spanish,es-noindentfirst]{babel}
\usepackage{graphicx}
%Code Highlighting
%He tenido que quitar el Minted porque me daba error
\usepackage{lastpage}

%TODO Establecer márgenes

%\usepackage{biblatex}
%\addbibresource{bibliografia.bib}

%Encabezados y pie de página
\usepackage{fancyhdr}               
\pagestyle{fancy}
\fancyhf{}
\fancyhead[L]{Plan de gestión, análisis, diseño y memorial del proyecto} %No cabe lo de Juego online <<Guiñote>>
\fancyfoot[L]{Grupo 11 | Susan L. Graham}
\fancyfoot[R]{\thepage \hspace{1pt} de \pageref{LastPage}}

\renewcommand{\headrulewidth}{2pt}
\renewcommand{\footrulewidth}{1pt}

\usepackage{hyperref}
\hypersetup{
    colorlinks=true,
    linkcolor=blue,
    filecolor=magenta,      
    urlcolor=cyan,
    pdftitle={Plan de gestión, análisis, diseño y memorial del proyecto <<Juego online "Guiñote">>},                   
    bookmarks=true,
}


\title{Plan de gestión, análisis, diseño y memorial del proyecto}                            
\author{Javier Herrer Torres}      
\date{Febrero 2021}                            

\begin{document}

\begin{titlepage}
    \begin{center}
        \vspace*{1cm}
            
        \Huge
        \textbf{Juego online <<Guiñote>>}
            
        \vspace{0.5cm}
        \LARGE
		Plan de gestión, análisis, diseño y memorial del proyecto
		
        \vspace{1.5cm}
            
        \textbf{Mario Chavanel Moreno} (NIP: 679551)\\
        \textbf{Javier Fuster Trallero} (NIP: 626901)\\
        \textbf{Daniel Gracia Pardo} (NIP: 756128)\\
        \textbf{Javier Herrer Torres} (NIP: 776609)\\
        \textbf{Daniel Pérez Ramírez} (NIP: 756558)\\
        \textbf{Samuel Torres Fau} (NIP: 780505)
        
            
        \vfill
            
        Grupo 11 | Susan L. Graham\\
        \href{https://github.com/UNIZAR-30226-2021-11}{https://github.com/UNIZAR-30226-2021-11}
            
        \vspace{1.5cm}
            
        %Cambiar logotipo EINA por logo identificativo
        \includegraphics[width=0.6\textwidth]{./images/logo.png}
            
        \vspace{1.5cm}
 
    \end{center}
\end{titlepage}

\tableofcontents

\section{Introducción}
%Resumen del proyecto, propósito, alcance, objetivos, entregables e hitos principales. Alrededor de una página es suficiente. No olvidar una breve descripción de la estructura del resto del documento.

\section{Organización del proyecto}
%Equipo del proyecto: integrantes del mismo, roles y responsabilidades. Qué hace dentro del proyecto cada miembro del equipo. Aunque es normal que todo el mundo haga varias cosas, también es importante que haya responsables definidos para las tareas importantes. Es importante designar a un director o directora de proyecto.
El equipo encargado del proyecto está compuesto por Javier Herrer Torres, Javier Jesús Fuster Trallero, Daniel Pérez Ramírez, Samuel Torres Fau, Mario Chavanel Moreno y Daniel Gracia Pardo. \\A su vez, se han creado tres grupos de trabajo para cada una de las 3 principales partes de las que consta el proyecto:
\subsection{Backend}
Este grupo está compuesto por Javier Herrer y Javier Fuster. El objetivo principal de este grupo es el desarrollo del backend, el sistema que se encarga de gestionar las partidas de guiñote (interfaces, lógica de juego, etc). Las aplicaciones realizadas por los otros grupos se conectarán a este sistema, que ofrecerá las distintas funcionalidades indicadas en el análisis de requisitos presente en este mismo documento.

\subsection{Aplicación Web}
Este grupo está compuesto por Mario Chavanel y Daniel Gracia, y tiene el objetivo de desarrollar el cliente web que permitirá jugar al guiñote desde navegador, conectándose al backend desarrollado por el primer grupo.

\subsection{Aplicación android}
Grupo compuesto por Daniel Pérez y Samuel Torres, con la finalidad de desarrollar la aplicación android que, al igual que la aplicación web, se conectará al backend para permitir jugar desde los smartphones con sistema operativo android.

Pese a que estén definidos los grupos y los integrantes de cada uno de estos, dado que la carga de trabajo necesaria para el desarrollo de alguna de las partes del sisteam es mayor en ciertos grupos que en otros, en algunos momentos se podrá apoyar en el desarrollo de componentes de grupos ajenos.

\subsection{Director del proyecto}
El director del proyecto elegido por el grupo es Javier Herrer, mediante una votación improvisada con resultado unánime.



\section{Plan de gestión del proyecto}
\subsection{Procesos}
%En esta sección se describe cómo se llevarán a cabo distintas tareas que hay que realizar en distintos momentos del proyecto.
\subsubsection{Procesos de inicio}
%Cómo se van a identificar y asignar recursos (p.ej. conseguir servidores en cloud o teléfonos móviles para pruebas, pero también registrarse para acceder a API que se quieran integrar o a herramientas online que se quieran usar etc.).

%Cómo se va a abordar la formación inicial de los miembros del equipo (revisar qué tecnologías se van a usar, qué componentes se van a integrar, con qué API hay que conectar y quiénes tienen que formarse, o autoformarse, en todas esas cosas y de qué manera (hacer algún curso online, planificar algo de tiempo para auto-formación con tutoriales y documentación etc.).

\subsubsection{Procesos de ejecución y control}
%Cómo se llevarán a cabo las comunicaciones internas, el registro de las decisiones tomadas en reuniones, la redacción de las actas etc.

%Cómo se van a determinar las tareas a realizar y el reparto de las mismas a integrantes del equipo en el día a día.

%Cómo se abordarán los temas de gestión del equipo humano (p.ej. la resolución de disputas).

%Qué se va a hacer respecto a medidas de progreso y monitorización del estado del proyecto (qué se mira/mide, cada cuánto tiempo, qué se hace si se detectan problemas de rendimiento o avance insuficiente o desviaciones respecto al plan inicial...).

%Cómo se hará la entrega de resultados.

\subsubsection{Procesos técnicos}
%Describir los métodos, herramientas y técnicas necesarios tanto para construir el software (p.ej. herramientas de desarrollo) como para desplegarlo y probarlo (todos los necesarios para dar soporte a los planes descritos a continuación).

\subsection{Planes}
%Un plan establece un objetivo y, en general, qué necesitamos para conseguirlo. Llevar a cabo los planes que se describan aquí requerirá aplicar los procesos descritos anteriormente, algunos una vez y otros muchas veces. Un plan suele incluir cuándo se llevan a cabo estos procesos (periódicamente, o en fechas concretas, o “al menos N veces” etc.) y quién es responsable (personas concretas a veces, pero generalmente roles) de hacerlos, o de asegurarse que se hacen.

\subsubsection{Plan de gestión de configuraciones}
%Convenciones de nombres (documentos) y estándares de código (guías de estilos).

%Responsable o responsables de las distintas actividades (puesta en marcha, apoyo al equipo, revisión de commits, copias de seguridad, control de las versiones entregadas a cliente...).

%Recursos: repositorios de control de versiones (cuáles, cuántos, permisos de acceso a los mismos) y sistema de gestión de incidencias.

%Procedimiento para realizar cambios al código fuente y los documentos técnicos: workflow de control de versiones utilizado, cuándo/cómo se permiten realizar commits al repositorio compartido, si tienen que ser aceptados por alguien previamente o no, qué hay que anotar en el sistema de gestión de incidencias, quién decide el estado de las incidencias, en qué estados puede estar una incidencia etc.

\subsubsection{Plan de construcción y despliegue del software}
%Cómo se construye e integra el software: si hay scripts de construcción automatizada o no (en ese caso qué se usa, y cómo se garantiza que todos los participantes compilan igual y con las mismas dependencias), qué se incluye en la construcción (descarga y actualización de dependencias, compilación, ejecución de tests automáticos...) y cada cuánto se construye (compila, integra, prueba) el sistema completo, cómo se configuran los computadores de los desarrolladores.

%Cómo se despliega el software más allá de las máquinas de desarrollo: contenedores, máquinas virtuales, servidor en cloud etc. y cómo se configuran esos entornos (rutas, usuarios y contraseñas, puertos y otros elementos).

\subsubsection{Calendario del proyecto y división del trabajo}
%Diagrama de Gantt que recoja las tareas a realizar. Tened en cuenta que trabajáis con dos iteraciones y por tanto que hay una entrega intermedia y una final, y esto debe estar reflejado en este diagrama. Tened en cuenta que es normal que lo tengáis que actualizar conforme avance el proyecto (cuándo y cómo establezcáis en la sección 3.1.2 (Procesos de ejecución y control).
	%Debe quedar claro qué requisitos van a estar completados en la primera iteración y cuáles en la segunda. Es posible que para la primera iteración no se planifique completar ningún requisito, pero en ese caso tiene que planificarse qué se hará y que faltará por hacer para cada requisito.

%División del trabajo en partes (los módulos del software a desarrollar, pero también la documentación, el diseño gráfico, instalaciones o despliegues, pruebas manuales etc.) y reparto de los mismos entre el equipo de desarrollo, al menos a alto nivel (el reparto de labores concretas en el día a día no se detalla aquí, pero hay que explicar bajo qué criterios y quién/cómo se hace eso en la sección 3.1.2 (Procesos de ejecución y control). Debe haber una correspondencia con las tareas que aparecen en el diagrama de Gantt (que no necesariamente tiene que ser una relación uno a uno).
	%Verificad que esta división del trabajo cubre todos los requisitos.

\section{Análisis y diseño del sistema}
\subsection{Análisis de requisitos}
%Hay que completar y detallar los requisitos preliminares incluidos en la propuesta técnica y económica. Recordad que los requisitos deben ser completos, concretos, medibles cuando tenga sentido y lo menos ambiguos posible. También es importante que estén identificados para facilitar su trazabilidad.

\subsection{Diseño del sistema}
%Diagramas arquitecturales (de módulos, de componentes y conectores, de distribución), patrones de diseño y estilos arquitecturales que se aplicarán. Las interfaces (de módulos y de componentes) son especialmente importantes. También lo son los protocolos de comunicación entre componentes.

%Tecnologías elegidas (lenguajes de programación, componentes que se integrarán, API web externas con las que se conectará etc.).

%Otros aspectos técnicos de interés (p.ej. si hay base de datos si va a ser SQL o NoSQL, si hay una API Web va a ser REST[ful] o no, si algunas de las operaciones van a ser asíncronas o no, si va a ser una aplicación móvil o de escritorio será nativa o se van a usar tecnologías web, cómo se van a considerar los requisitos de seguridad o de prestaciones, cómo y dónde se harán las instalaciones y despliegues etc.)

%Hay que justificar todas las decisiones de diseño. Esto exige contestar a dos preguntas sobre cada decisión: ¿qué alternativas se barajaron? y ¿por qué se eligió una y no las otras?


\end{document}
